\documentclass{article}
\usepackage{amsmath}
\usepackage{fullpage}
\usepackage{logicproof}
\usepackage{parskip}
\usepackage{forest}
\usepackage{float}
\usepackage{hyperref}
\usepackage{colortbl, xcolor}

\definecolor{Red}{rgb}{1,0.8,0.8}

\newcommand{\imp}{\ensuremath{\rightarrow}}
\newcommand{\seq}{\ensuremath{\vdash}}
\newcommand{\ent}{\ensuremath{\models}}
\newcommand{\elim}{\ensuremath{\mathit{e}}}
\newcommand{\intr}{\ensuremath{\mathit{i}}}
\newcommand{\rep}[1]{copy #1}

% p116
\newcommand{\landi}[2]{$\land_\intr$ #1, #2}
\newcommand{\landex}[1]{$\land_{\elim_1}$ #1}
\newcommand{\landey}[1]{$\land_{\elim_2}$ #1}
\newcommand{\lorix}[1]{$\lor_{\intr_1}$ #1}
\newcommand{\loriy}[1]{$\lor_{\intr_2}$ #1}
\newcommand{\lore}[5]{$\lor_\elim$ #1, #2--#3, #4--#5}

% p117
\newcommand{\impi}[2]{$\imp_\intr$ #1--#2}
\newcommand{\impe}[2]{$\imp_\elim$ #1, #2}

% p118
\newcommand{\negi}[2]{$\neg_\intr$ #1--#2}
\newcommand{\nege}[2]{$\neg_\elim$ #1, #2}

% p119
\newcommand{\bote}[1]{$\bot_\elim$ #1}
\newcommand{\nnege}[1]{$\neg\neg_\elim$ #1}

% p120
\newcommand{\nnegi}[1]{$\neg\neg_\intr$ #1}

% 122
\newcommand{\modt}[2]{$\mathit{MT}$ #1, #2}

% 123
\newcommand{\pbc}[2]{$\mathit{PBC}$ #1--#2}

% 124
\newcommand{\lem}{\ensuremath{\mathit{LEM}}}

% formula logic
\newcommand{\fneg}[1]{\ensuremath{\neg \left( #1 \right)}}
\newcommand{\fland}[2]{\ensuremath{\left( #1 \right) \land \left( #2 \right)}}
\newcommand{\flor}[2]{\ensuremath{\left( #1 \right) \lor \left( #2 \right)}}
\newcommand{\fimp}[2]{\ensuremath{\left( #1 \right) \imp \left( #2 \right)}}

% formula logic, parenthesis override
\newcommand{\Fneg}[1]{\ensuremath{\neg #1}}
\newcommand{\FlanD}[2]{\ensuremath{#1 \land #2}}
\newcommand{\Fland}[2]{\ensuremath{#1 \land \left( #2 \right)}}
\newcommand{\flanD}[2]{\ensuremath{\left( #1 \right) \land #2}}
\newcommand{\FloR}[2]{\ensuremath{#1 \lor #2}}
\newcommand{\Flor}[2]{\ensuremath{#1 \lor \left( #2 \right)}}
\newcommand{\floR}[2]{\ensuremath{\left( #1 \right) \lor #2}}
\newcommand{\FimP}[2]{\ensuremath{#1 \imp #2}}
\newcommand{\Fimp}[2]{\ensuremath{#1 \imp \left( #2 \right)}}
\newcommand{\fimP}[2]{\ensuremath{\left( #1 \right) \imp #2}}

\edef\restoreparindent{\parindent=\the\parindent\relax}
\usepackage{parskip}
\restoreparindent

\title{CMPE 58S: Sp. Tp. Computer Aided Verification \\ Homework II: Some Exercises}
\date{2018 October 5}
\author{Mehmet Utkan Gezer \\ 2018700060}

\begin{document}
\maketitle

\section{Exercises from book}
Following exercises are from the book ``Logic in Computer Science:
Modelling and Reasoning about Systems'' by Michael Huth and Mark Ryan.

\subsection{Exercise 1.2.1.x}   % Page 95
Validity of the following is to be proven.
$$
p \imp (q \lor r), q \imp s, r \imp s \seq p \imp s
$$

Proof:
\begin{logicproof}{2}
    p \imp (q \lor r)   & premise\\
    q \imp s            & premise\\
    r \imp s            & premise\\
    \begin{subproof}
        p               & assumption\\
        q \lor r        & \impe{1}{4}\\
        \begin{subproof}
            q           & assumption\\
            s           & \impe{2}{6}
        \end{subproof}
        \begin{subproof}
            r           & assumption\\
            s           & \impe{3}{8}
        \end{subproof}
        s               & \lore{5}{6}{7}{8}{9}
    \end{subproof}
    p \imp s            & \impi{4}{10}
\end{logicproof}

\pagebreak
\subsection{Exercise 1.2.2.d}   % Page 95
Validity of the following is questioned.
We claim it to be true, and prove it.
$$
p \lor q, \neg q \lor r \seq p \lor r
$$

Proof:
\begin{logicproof}{2}
    p \lor q                & premise\\
    \neg q \lor r           & premise\\
    q \lor \neg q           & \lem\\
    \begin{subproof}
        q                   & assumption\\
        \begin{subproof}
            \neg q          & assumption\\
            \bot            & \nege{4}{5}\\
            r               & \bote{6}
        \end{subproof}
        \begin{subproof}
            r               & assumption
        \end{subproof}
        r                   & \lore{2}{5}{7}{8}{8}\\
        p \lor r            & \loriy{9}
    \end{subproof}
    \begin{subproof}
        \neg q              & assumption\\
        \begin{subproof}
            p               & assumption
        \end{subproof}
        \begin{subproof}
            q               & assumption\\
            \bot            & \nege{13}{11}\\
            p               & \bote{14}
        \end{subproof}
        p                   & \lore{1}{12}{12}{13}{15}\\
        p \lor r            & \lorix{16}
    \end{subproof}
    p \lor r                & \lore{3}{4}{10}{11}{17}
\end{logicproof}

\pagebreak
\subsection{Exercise 1.2.3.q}   % Page 96
Validity of the following is to be proven using \lem.
$$
\seq (p \imp q) \lor (q \imp r)
$$

Proof:
\begin{logicproof}{2}
    q \lor \neg q                   & \lem\\
    \begin{subproof}
        q                           & assumption\\
        \begin{subproof}
            p                       & assumption\\
            q                       & \rep{2}
        \end{subproof}
        p \imp q                    & \impi{3}{4}\\
        (p \imp q) \lor (q \imp r)  & \lorix{5}
    \end{subproof}
    \begin{subproof}
        \neg q                      & assumption\\
        \begin{subproof}
            q                       & assumption\\
            \bot                    & \nege{7}{8}\\
            r                       & \bote{9}
        \end{subproof}
        q \imp r                    & \impi{8}{10}\\
        (p \imp q) \lor (q \imp r)  & \loriy{11}
    \end{subproof}
    (p \imp q) \lor (q \imp r)      & \lore{1}{2}{6}{7}{12}
\end{logicproof}

\subsection{Exercise 1.3.4.b}   % Page 97
Parse tree of the following formula is to be drawn:
$$
\floR{\FimP{\flor{\FimP{p}{\Fneg{q}}}{\FlanD{p}{r}}}{s}}{\Fneg{r}}
$$

The parse tree can be seen in Figure~\ref{fig:ex134b}.
\begin{figure}[H]
    \centering
    \begin{forest}
    for tree={circle, draw, l sep=10pt, s sep=20pt}
    [$\lor$[$\imp$[$\lor$[$\imp$[$p$][$\neg$[$q$]]][$\land$[$p$][$r$]]][$s$]][$\neg$[$r$]]]
    \end{forest}
    \caption{Parse tree for the formula.}
    \label{fig:ex134b}
\end{figure}

\subsection{Exercise 1.4.5} % Page 101
Validity and satisfiability of the formula of
the parse tree in Figure 1.10 on page 44 is asked.
You can see the redrawn parse tree on Figure~\ref{fig:ex145}.

\begin{figure}[H]
    \centering
    \begin{forest}
    for tree={circle, draw, l sep=10pt, s sep=20pt}
    [$\neg$[$\land$[$\imp$[$q$][$\neg$[$p$]]][$\imp$[$p$][$\lor$[$r$][$q$]]]]]
    \end{forest}
    \caption{Redrawn parse tree of Figure 1.10 on page 44.}
    \label{fig:ex145}
\end{figure}

The formula of the parse tree is as follows:
$$
\fneg{\fland{\FimP{q}{\Fneg{p}}}{\Fimp{p}{\FloR{r}{q}}}}
$$

Using the semantic equivalence rules
(elimination of implication, De Morgan's laws and
elimination of double negation),
we can transform this formula as follows:
\begin{align*}
&\equiv \fneg{\fland{\FloR{\Fneg{q}}{\Fneg{p}}}{\Flor{\Fneg{p}}{\FloR{r}{q}}}}%
&&\text{(elimination of implication)}\\
&\equiv \fneg{\fland{\FloR{\Fneg{q}}{\Fneg{p}}}{\Fneg{p} \lor r \lor q}}%
&&\text{(elimination of parenthesis)}\\
&\equiv \FloR{\fneg{\FloR{\Fneg{q}}{\Fneg{p}}}}{\fneg{\Fneg{p} \lor r \lor q}}%
&&\text{(distribution of negation)}\\
&\equiv \flor{\FlanD{\Fneg{\Fneg{q}}}{\Fneg{\Fneg{p}}}}{\Fneg{\Fneg{p}} \land \Fneg{r} \land \Fneg{q}}%
&&\text{(distribution of negation)}\\
&\equiv \flor{\FlanD{q}{p}}{p \land \Fneg{r} \land \Fneg{q}}%
&&\text{(elimination of double negation)}
\end{align*}

This is the DNF (Disjunctive Normal Form) of the initial formula.
We then build the following partial truth table for the formula:

$$
\begin{array}{c|c|c||c|c|c}
p      & q      & r      & \FlanD{q}{p} & p \land \Fneg{r} \land \Fneg{q} & \flor{\FlanD{q}{p}}{p \land \Fneg{r} \land \Fneg{q}}\\\hline
T      & T      & T      & T            & F                               & T\\
\vdots & \vdots & \vdots & \vdots       & \vdots                          & \vdots\\
F      & T      & T      & F            & F                               & F\\
\vdots & \vdots & \vdots & \vdots       & \vdots                          & \vdots
\end{array}
$$

With only those two lines of the truth table for the
semantically equivalent formula
$\flor{\FlanD{q}{p}}{p \land \Fneg{r} \land \Fneg{q}}$
we can say that
$\fneg{\fland{\FimP{q}{\Fneg{p}}}{\Fimp{p}{\FloR{r}{q}}}}$
is satisfiable, but not valid:
\begin{itemize}
\item
\textbf{It is satisfiable}, because there exists an evaluation,
namely $(p, q, r) = (T, T, T)$ for which
the formula evaluates as $T$.
\item
\textbf{It is not valid}, and the proof for it is by contradiction.
When we assume it is valid, it should follow that
for all evaluations of the atoms the formula should
evaluate to $T$. However, for the evaluation
$(p, q, r) = (F, T, T)$, the formula evaluates as $F$,
which is a contradiction.
\end{itemize}


\subsection{Exercise 1.5.4} % Page 103
Soundness or completeness is to be used to show that
a sequent $\phi_1, \phi_2, \dotsc, \phi_n \seq \psi$
has a proof iff
$\phi_1 \imp \phi_2 \imp \dotsb \imp \phi_n \imp \psi$
is a tautology.

The statement is not even correct, and we can show that using
soundness and completeness and a proof by contradiction.

Assume that the statement is true. The statement should be true
for all $n$, so let $n = 2$. We then have a statement saying
a sequent $\phi_1, \phi_2 \seq \psi$
has a proof iff $\phi_1 \imp \phi_2 \imp \psi$
is a tautology.

By Remark 1.12, we can transform
the original sequent $\phi_1, \phi_2 \seq \psi$
as follows: $\seq \phi_1 \imp (\phi_2 \imp \psi)$.
Let us now transform them both using semantic equivalence rules:
\begin{align*}
\phi_1 \imp (\phi_2 \imp \psi) &\equiv \neg\phi_1 \lor (\neg\phi_2 \lor \psi)  &  \phi_1 \imp \phi_2 \imp \psi &\equiv \neg\phi_1 \lor \phi_2 \imp \psi\\
                               &\equiv \neg\phi_1 \lor \neg\phi_2 \lor \psi    &                               &\equiv \neg(\neg\phi_1 \lor \phi_2) \lor \psi\\
                               &                                               &                               &\equiv (\neg\neg\phi_1 \land \neg\phi_2) \lor \psi\\
                               &                                               &                               &\equiv (\phi_1 \land \neg\phi_2) \lor \psi\\
                               &                                               &                               &\equiv (\phi_1 \lor \psi) \land (\neg\phi_2 \lor \psi)
\end{align*}

By the statement assumed to be true, if the original sequent,
or equivalently its transformed form, or equivalently
its semantic equivalent has a proof, then the second formula,
or its semantic equivalent should be a tautology. Let us now
prepare their truth tables:
$$
\begin{array}{c|c|c||c|c}
\phi_1 & \phi_2 & \psi & \neg \phi_1 \lor \neg \phi_2 \lor \psi & (\phi_1 \lor \psi) \land (\neg\phi_2 \lor \psi) \\\hline
T      & T      & T    & T                                      & T\\
T      & T      & F    & F                                      & F\\
T      & F      & T    & T                                      & T\\
T      & F      & F    & T                                      & T\\
F      & T      & T    & T                                      & T\\
\rowcolor{Red}
F      & T      & F    & T                                      & F\\
F      & F      & T    & T                                      & T\\
\rowcolor{Red}
F      & F      & F    & T                                      & F
\end{array}
$$

On the truth table, we can clearly see that there are
cases (highlighted in red) where sequent has a proof,
yet the formula is not a tautology.
This is a contradiction, meaning that our assumption
was wrong. Statement that a sequent
$\phi_1, \phi_2, \dotsc, \phi_n \seq \psi$ has a proof iff
$\phi_1 \imp \phi_2 \imp \dotsb \imp \phi_n \imp \psi$
is a tautology, is not true.

\section{Classification of formulas}
Following formulas are to be classified as valid, satisfiable, or
not satisfiable, using the semantic method.

\subsection{An implication}
The formula is as follows:
$$
\neg q \imp q
$$

We can transform it into the CNF form:
\begin{align*}
&\equiv \neg\neg q \lor q\\
&\equiv q \lor q
\end{align*}

By Lemma 1.43, a disjunction of literals
$L_1 \lor L_2 \lor \dotsb \lor L_m$ is valid iff there are
$1 \leq i, j \leq m$ such that $L_i$ is $\neg L_j$.
Since $q \lor q$ has no such pair of literals,
this disjunction of literals is \textbf{not valid}.

Proposition 1.45 states that a formula is satisfiable iff
its negation is not valid. Negation of our formula is
$\fneg{q \lor q}$, and in CNF form:
\begin{align*}
&\equiv \neg q \land \neg q
\end{align*}

The resulting formula is the conjunction of two identical
disjunction of literals that consist of a single literal.
Since there is no pair of literals of form $L_i = \neg L_j$ in
$\neg q$, the negated formula is not valid by Lemma 1.43, and therefore
our original formula is \textbf{satisfiable} by Proposition 1.45.

\subsection{A conjunction}
The formula is as follows:
$$
\fland{\neg q \imp q}{q \imp \neg q}
$$

We can transform it into the CNF form:
\begin{align*}
&\equiv \fland{\neg\neg q \lor q}{\neg q \lor \neg q}\\
&\equiv \fland{q \lor q}{\neg q \lor \neg q}
\end{align*}

For the formula to be valid, both the $q \lor q$ and
$\neg q \lor \neg q$ needs to be valid. However, as proven in
the previous exercise, former is not valid, therefore the
formula is also \textbf{not valid}.

The negation of the formula is:
$$
\fneg{\fland{q \lor q}{\neg q \lor \neg q}}.
$$

We can transform it into the CNF form:
\begin{align*}
&\equiv \FloR{\fneg{q \lor q}}{\fneg{\neg q \lor \neg q}}\\
&\equiv \flor{\neg q \land \neg q}{\neg\neg q \land \neg\neg q}\\
&\equiv \flor{\neg q \land \neg q}{q \land q}\\
&\equiv \fland{\floR{\neg q \land \neg q}{q}}{\floR{\neg q \land \neg q}{q}}\\
&\equiv \fland{\fland{\neg q \lor q}{\neg q \lor q}}{\fland{\neg q \lor q}{\neg q \lor q}}\\
&\equiv \FlanD{\fland{\neg q \lor q}{\neg q \lor q}}{\fland{\neg q \lor q}{\neg q \lor q}}
\end{align*}

The resulting formula is the conjunction of 4 identical
disjunction of literals that consist of $q$ and $\neg q$.
Since $L_i = q$ and $L_j = \neg q$ is of the form $L_i = \neg L_j$
it is valid, and therefore all 4 of them are valid, and consequently
the negated formula is also valid.

By Proposition 1.45, negated formula being valid means that
the original formula is \textbf{not satisfiable}.

\end{document}
